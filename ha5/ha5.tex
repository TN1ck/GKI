\documentclass[a4paper,10pt]{article}
\usepackage[utf8]{inputenc}
\usepackage{amsmath}
\usepackage{amsfonts}
\usepackage{amssymb}
\usepackage[german]{babel}
\setlength{\parindent}{0cm}
\usepackage{setspace}
\usepackage{mathpazo}
\usepackage{listings}
\usepackage{graphicx}
\usepackage{wasysym}
\usepackage{booktabs}
\usepackage{verbatim}
\usepackage{ulem}
\usepackage{enumerate}
\usepackage{hyperref}
\usepackage{ulem}
\usepackage{stmaryrd }
\usepackage[a4paper,
left=1.8cm, right=1.8cm,
top=2.0cm, bottom=2.0cm]{geometry}
\usepackage{tabularx}
\usepackage{tikz}
\usetikzlibrary{trees,petri,decorations,arrows,automata,shapes,shadows,positioning,plotmarks}

\newcommand{\rf}{\right\rfloor}
\newcommand{\lf}{\left\lfloor}
\newcommand{\tabspace}{15cm}
\newcommand{\N}{\mathbb{N}}
\newcommand{\Z}{\mathbb{Z}}

\begin{document}
\begin{center}
\Large{Grundlagen der künstlichen Intelligenz: Hausaufgabe 5} \\
\end{center}
\begin{tabbing}
Tom Nick \hspace{2cm}\= - 340528\\
Niklas Gebauer \> - 340942 \\
Leonard Witte \> - 341457 \\
Johannes Herrmann \> - 341091\\
\end{tabbing}

\section*{Aufgabe 1}
    Wir definieren folgende Ereignisse:
    \begin{align*}
        I_c &:= \text{Das Auto hat die Farbe $c \in \{B,G\}$} \\
        E_c &:= \text{Das Auto erscheint in der Farbe $c \in \{B, G\}$}
    \end{align*}
    Aus dem Text kennen wir folgende Wahrscheinlichkeiten:
    \begin{align*}
        P(E_B \mid I_B) &= 0.7 \\
        P(E_G \mid I_B) &= 0.3
    \end{align*}
    \begin{enumerate}[~~a.)]
        \item Nach Bayes wäre die Rechnung:
        \begin{align}
            P(I_B \mid E_B) = \frac{P(E_B\mid P_B) \cdot P(I_B)}{P(E_B)}
        \end{align}
        Es ist leicht zu sehen, dass die Wahrscheinlichkeit für $P(I_B)$ bzw. $P(I_G)$ benötigt werden, die wir aber nicht kennen, somit können wir mit den derzeitigen Informationen nicht die wahrscheinlichste Farbe des Autos berechnen.
        \item Nun kennen wir:
        \begin{align*}
            P(I_B) &= 0.2 \\
            P(I_G) &= 0.8
        \end{align*}
        Aus $(1)$ folgt:
        \begin{align*}
            P(I_B \mid E_B) &= \frac{P(E_B\mid P_B) \cdot P(I_B)}{P(E_B \mid I_B) \cdot P(I_B) + P(E_B \mid I_G) \cdot P(I_G)} \\
            &= \frac{0.7 \cdot 0.2}{0.7 \cdot 0.2 + 0.3 \cdot 0.8} \\
            &= \frac{0.14}{0.14 + 0.24} = \frac{0.14}{0.38} = 0.368
        \end{align*}
        Die Gegenwahrscheinlichkeit $P(I_G \mid E_B)$ ist damit $0.632$, womit es fast doppelt so Wahrscheinlich ist, dass die Person ein Grünes, anstatt ein Blaues Taxi gesehen hatte.
    \end{enumerate}
\section*{Aufgabe 2}
\section*{Aufgabe 3}
\section*{Aufgabe 4}
    \begin{enumerate}[~~a.)]
        \item
        \item
        \item
        \item
    \end{enumerate}
\end{document}
