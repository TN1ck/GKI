\documentclass[a4paper,10pt]{article}
\usepackage[utf8]{inputenc}
\usepackage{amsmath}
\usepackage{amsfonts}
\usepackage{amssymb}
\usepackage[german]{babel}
\setlength{\parindent}{0cm}
\usepackage{setspace}
\usepackage{mathpazo}
\usepackage{listings}
\usepackage{graphicx}
\usepackage{wasysym}
\usepackage{booktabs}
\usepackage{verbatim}
\usepackage{ulem}
\usepackage{enumerate}
\usepackage{hyperref}
\usepackage{ulem}
\usepackage{stmaryrd }
\usepackage[a4paper,
left=1.8cm, right=1.8cm,
top=2.0cm, bottom=2.0cm]{geometry}
\usepackage{tabularx}
\usepackage{tikz}
\usetikzlibrary{trees,petri,decorations,arrows,automata,shapes,shadows,positioning,plotmarks}

\newcommand{\rf}{\right\rfloor}
\newcommand{\lf}{\left\lfloor}
\newcommand{\tabspace}{15cm}
\newcommand{\N}{\mathbb{N}}
\newcommand{\Z}{\mathbb{Z}}

\begin{document}
\begin{center}
\Large{Grundlagen der künstlichen Intelligenz: Hausaufgabe 5} \\
\end{center}
\begin{tabbing}
Tom Nick \hspace{2cm}\= - 340528\\
Niklas Gebauer \> - 340942 \\
Leonard Witte \> - 341457 \\
Johannes Herrmann \> - 341091\\
\end{tabbing}

\section*{Aufgabe 1}
    Wir definieren folgende Ereignisse:
    \begin{align*}
        I_c &:= \text{Das Auto hat die Farbe $c \in \{B,G\}$} \\
        E_c &:= \text{Das Auto erscheint in der Farbe $c \in \{B, G\}$}
    \end{align*}
    Aus dem Text kennen wir folgende Wahrscheinlichkeiten:
    \begin{align*}
        P(E_B \mid I_B) &= 0.7 \\
        P(E_G \mid I_B) &= 0.3
    \end{align*}
    \begin{enumerate}[~~a.)]
        \item Nach Bayes wäre die Rechnung:
        \begin{align}
            P(I_B \mid E_B) = \frac{P(E_B\mid P_B) \cdot P(I_B)}{P(E_B)}
        \end{align}
        Es ist leicht zu sehen, dass die Wahrscheinlichkeit für $P(I_B)$ bzw. $P(I_G)$ benötigt werden, die wir aber nicht kennen, somit können wir mit den derzeitigen Informationen nicht die wahrscheinlichste Farbe des Autos berechnen.
        \item Nun kennen wir:
        \begin{align*}
            P(I_B) &= 0.2 \\
            P(I_G) &= 0.8
        \end{align*}
        Aus $(1)$ folgt:
        \begin{align*}
            P(I_B \mid E_B) &= \frac{P(E_B\mid P_B) \cdot P(I_B)}{P(E_B \mid I_B) \cdot P(I_B) + P(E_B \mid I_G) \cdot P(I_G)} \\
            &= \frac{0.7 \cdot 0.2}{0.7 \cdot 0.2 + 0.3 \cdot 0.8} \\
            &= \frac{0.14}{0.14 + 0.24} = \frac{0.14}{0.38} = 0.368
        \end{align*}
        Die Gegenwahrscheinlichkeit $P(I_G \mid E_B)$ ist damit $0.632$, womit es fast doppelt so wahrscheinlich ist, dass die Person ein grünes, anstatt ein blaues Taxi gesehen hatte.
    \end{enumerate}
\section*{Aufgabe 2}
	\begin{enumerate}[~~a.)]
	 \item
		\begin{align*}
			&P(X=x, E_{1} = e_{1}, E_{2} = e_{2}, ... , K_{1} = k_{1}, K_{2} = k_{2}, ... , C_{1} = c_{1}, C_{2} = c_{2}, ... , Z_{1} = z_{1}, Z_{2} = z_{2}, ...) \\
			&= P(X=x | E_{1} = e_{1}, E_{2} = e_{2}, ...) *
			\prod_{i} P(E_{i}=e_{i} | eltern(E_{i})) *
			\prod_{j} P(K_{j}=k_{j} | eltern(K_{j})) \\
			&* \prod_{l} P(C_{l}=c_{l} | eltern(C_{l})) *
			\prod_{m} P(Z_{m}=z_{m} | eltern(Z_{m}))\\
		\end{align*}
	 	mit:
	 	\begin{align*}
			eltern(E_{i}) &\subseteq \{C_{1} = c_{1}, C_{2} = c_{2}, ... , Z_{1} = z_{1}, Z_{2} = z_{2}, ...\} \\
			eltern(K_{j}) &\subseteq \{X = x, C_{1} = c_{1}, C_{2} = c_{2}, ...\} \text{und} X \in eltern(K_{j}) \\
			eltern(C_{l}) &\subseteq \{E_{1} = e_{1}, E_{2} = e_{2}, ... , K_{1} = k_{1}, K_{2} = k_{2}, ... , C_{1} = c_{1}, C_{2} = c_{2}, ... , Z_{1} = z_{1}, Z_{2} = z_{2}, ...\} \\
			eltern(Z_{m}) &\subseteq \{E_{1} = e_{1}, E_{2} = e_{2}, ... , K_{1} = k_{1}, K_{2} = k_{2}, ... , C_{1} = c_{1}, C_{2} = c_{2}, ... , Z_{1} = z_{1}, Z_{2} = z_{2}, ...\} \\
		\end{align*}	
	 \item
	 Wir nummerieren die einzelnen Produkte aus a) zur besseren Lesbarkeit des Beweises wie folgt:
	 \begin{align*}
	 	&(1) P(X=x | E_{1} = e_{1}, E_{2} = e_{2}, ...) \\
	 	&(2) \prod_{i} P(E_{i}=e_{i} | eltern(E_{i})) \\
	 	&(3) \prod_{j} P(K_{j}=k_{j} | eltern(K_{j}))\\
	 	&(4) \prod_{l} P(C_{l}=c_{l} | eltern(C_{l}))\\
	 	&(5) \prod_{m} P(Z_{m}=z_{m} | eltern(Z_{m}))\\
	 \end{align*}
	 	Es gilt:
	 	\begin{align*}
	 	&P(X=x | mb(X), Z_{1} = z_{1}, Z_{2} = z_{2}, ...) \\
	 	&\stackrel{\text{Def. bed. Wkt.}}= \frac{P(X=x , mb(X), Z_{1} = z_{1}, Z_{2} = z_{2}, ...)}{\sum_{x} P(X = x , mb(X), Z_{1} = z_{1}, Z_{2} = z_{2}, ...)}\\
	 	&\stackrel{\text{Fakt. Verb.-Wkt.}}= \frac{(1)*(2)*(3)*(4)*(5)}{\sum_{x}[(1)*(2)*(3)*(4)*(5)]} \\
	 	&\stackrel{\text{(2),(4),(5) unabh. v. X}}= \frac{(1)*(2)*(3)*(4)*(5)}{(2)*(4)*(5)*\sum_{x}[(1)*(3)]} \\
	 	&= \frac{\sum_{z_{1}, z_{2},...}[(2)*(4)*(5)]*(1)*(3)}{\sum_{z_{1}, z_{2},...}[(2)*(4)*(5)]*\sum_{x}[(1)*(3)]} \\	
	 	&\stackrel{\text{(1),(3) unabh. v. } Z_{m}}= \frac{\sum_{z_{1}, z_{2},...}[(2)*(4)*(5)*(1)*(3)]}{\sum_{z_{1}, z_{2},...}[(2)*(4)*(5)*\sum_{x}[(1)*(3)]]} \\
	 	&\stackrel{\text{(2),(4),(5) unabh. v. X}}= \frac{\sum_{z_{1}, z_{2},...}[(2)*(4)*(5)*(1)*(3)]}{\sum_{x,z_{1}, z_{2},...}[(2)*(4)*(5)*(1)*(3)]} \\
	 	&\stackrel{\text{Fakt. Verb.-Wkt.}}= \frac{P(X = x, mb(X))}{mb(X)} \\
	 	&\stackrel{\text{Def. bed. Wkt.}}=P(X = x | mb(X))
		\end{align*}	 	 			
	\end{enumerate}
\section*{Aufgabe 3}
Sei
\begin{align*}
    X &:= \text{Das Produkt der Augenzahlen der zwei Würfel} \\
    Y &:= \text{Die Augenzahl eines Würfel}
\end{align*}
Das Spiel ist fair, wenn der Erwartungswert von $X$ gleich dem Einsatz ist, dann ist der Gewinn/Verlust der Spieler 0euro. Es ist offensichtlich das $X$ sich ergibt, wenn die Erwartungswerte der Augenzahl beider Würfel multipliziert wird und dass das Werfen zweier Würfel unabhängige Ereignisse sind, somit ergibt sich:
\begin{align*}
    E(X) = E(Y*Y) \\
    E(X) = E(Y) \cdot E(Y) \\
    E(X) = 3.5 \cdot 3.5 \\
    E(X) = 12.25
\end{align*}
Also muss der Einsatz 12.25euro betragen, damit das Spiel fair ist.
\section*{Aufgabe 4}
    \begin{enumerate}[~~a.)]
        \item ~\\
         \begin{center}
            \begin{tikzpicture}[auto,bend angle=30,node distance=2cm]
                % Zustaende
                \node[state] (1) {K};
                \node[state] (2)[below right of=1] {Z};
                \node[state] (3)[below left of=2] {W};
                \node[state] (4)[above right of=2] {F};
                \node[state] (5)[below right of=4] {C};

                \path[->] (1) edge  (2);
                \path[->] (3) edge  (2);
                \path[->] (2) edge  (5);
                \path[->] (1) edge  (4);
                \path[->] (4) edge  (5);
            \end{tikzpicture}
        \end{center}
        \item
            \begin{align*}
            %P(Z=f|K=w)&= \sum\limits_{x\in \{w,f\}}^{} P(Z=f|K=w,W=x) \cdot P(W=x)\\
            %&=P(Z=f|K=w,W=w)\cdot P(W=w)+P(Z=f|K=w,W=f)\cdot P(W=f)\\
            %&=0,9\cdot 0,4+0,5\cdot 0,6=0,36+0,3=0,66\\
            %P(C=w|K=w,Z=f)&=\frac{P(C=w,K=w,Z=f)}{P(K=w,Z=F)}\\
            P(C=w|K=w,Z=f)&=\sum\limits_{y\in \{w,f\}}^{} (P(C=w|F=y,Z=f)\cdot P(F=y|K=w))\\
            %&=\frac{\sum\limits_{y\in \{w,f\}}^{} (P(C=w|F=y,Z=f)\cdot P(F=y|K=w))\cdot P(K=w) \cdot P(Z=w|K=w)}{P(K=w) \cdot P(Z=w|K=w)}\\
            &=P(C=w|F=w,Z=f)\cdot P(F=w|K=w)+P(C=w|F=f,Z=f)\cdot P(F=f|K=w)\\
            %&=\frac{(0,9\cdot 0,6+0,1\cdot 0,4)\cdot 0,2\cdot 0,66}{0,2\cdot 0,66}=0,58
            &=0,9\cdot 0,6+0,1\cdot 0,4=0,58
            \end{align*}
        \item
            \begin{align*}
            P(K=w|F=w)&=\frac{P(K=w,F=w)}{P(F=w)}\\
            &=\frac{P(K=w)\cdot P(F=w|K=w)}{\sum\limits_{k\in \{w,f\}}^{} P(F=w|K=k)}\\
            &=\frac{P(K=w)\cdot P(F=w|K=w)}{P(F=w|K=w)+P(F=w|K=f)}\\
            &=\frac{0,2 \cdot 0,6}{0,6+0,1} \approx 0,1714
            \end{align*}
        \item
        \begin{align*}
        	&P(W = w | Z = w) \\
        	&= \frac{\sum_{F,K,C} [P(K) * P(W) * P(F |K) *P(Z|K,W)*P(C|F,Z)]}{\sum_{F,K,C,W} [P(K) * P(W) * P(F |K) *P(Z|K,W)*P(C|F,Z)]}\\
        	&= \frac{P(W) *\sum_{K} [P(K) *P(Z|K,W) * \sum_{F}[P(F |K) * \sum{C}[P(C|F,Z)]]]}{\sum_{W}[P(W) *\sum_{K} [P(K) *P(Z|K,W) * \sum_{F}[P(F |K) * \sum{C}[P(C|F,Z)]]]]}\\
        	&\stackrel{\sum_{F}=1,\sum_{C}=1}= \frac{P(W) *\sum_{K} [P(K) *P(Z|K,W)]}{\sum_{W}[P(W) *\sum_{K} [P(K) *P(Z|K,W)]]}\\
        	&= \frac{P(W = w) * [P(K =w)* P(Z = w|K=w,W=w)+P(K=f)*P(Z=w)|K=f,W=w)]}{\sum_{y \in \{w, f\}}P(W = y) * [P(K =w)* P(Z = w|K=w,W=w)+P(K=f)*P(Z=w)|K=f,W=w)]} \\
        	&= \frac{0,4* [0,2*0,9+0,8*0,5]}{0,4* [0,2*0,9+0,8*0,5] + 0,6 * [0,2*0,5+0,8*0,1]} \\
        	&= \frac{0,4*0,58}{0,4*0,58+0,6*0,18} \\
        	&\approx 0,68
        \end{align*}
    \end{enumerate}
\end{document}
