\documentclass[a4paper,10pt]{article}
\usepackage[utf8]{inputenc}
\usepackage{amsmath}
\usepackage{amsfonts}
\usepackage{amssymb}
\usepackage[german]{babel}
\setlength{\parindent}{0cm}
\usepackage{setspace}
\usepackage{mathpazo}
\usepackage{listings}
\usepackage{graphicx}
\usepackage{wasysym}
\usepackage{booktabs}
\usepackage{verbatim}
\usepackage{ulem}
\usepackage{enumerate}
\usepackage{hyperref}
\usepackage{ulem}
\usepackage{stmaryrd }
\usepackage[a4paper,
left=1.8cm, right=1.8cm,
top=2.0cm, bottom=2.0cm]{geometry}
\usepackage{tabularx}
\usepackage{tikz}
\usetikzlibrary{trees,petri,decorations,arrows,automata,shapes,shadows,positioning,plotmarks}

\newcommand{\rf}{\right\rfloor}
\newcommand{\lf}{\left\lfloor}
\newcommand{\tabspace}{15cm}
\newcommand{\N}{\mathbb{N}}
\newcommand{\Z}{\mathbb{Z}}

\begin{document}
\begin{center}
\Large{Grundlagen der künstlichen Intelligenz: Hausaufgabe 3} \\
\end{center}
\begin{tabbing}
Tom Nick \hspace{2cm}\= - 340528\\
Niklas Gebauer \> - 340942 \\
Leonard Witte \> - 341457 \\
Johannes Herrmann \> - 341091\\
\end{tabbing}

\section*{Aufgabe 1}

\begin{enumerate}[~~a)]
    \item $\forall x (Stack(x) \Rightarrow ((\exists y (Block(y) \land in(y,x))) \land (\forall y(in(y,x) \Rightarrow Block(y)))))$

    \item $\forall x (Stack(x) \Rightarrow \exists y ((Block(y) \land in(y,x) \land on(y,T)) \land \forall z ((Block(z) \land in(z,x) \land on(z,T)) \Rightarrow z=y)))$

    \item $\forall x (Block(x) \Rightarrow ( (\lnot\exists y ((Block(y) \land on(x,y)) \land on(x,T)) \lor (\exists y (Block(y) \land on(x,y)) \land \lnot on(x,T) )))$

    \item $\forall x (Block(x) \land \lnot \exists y (Block(y) \land on(y,x))) \Rightarrow top(x)$

    \item $\forall x \forall y (Block(x) \land Block(y) \land ((on(x,y) \land \lnot\exists z (Stack(z) \land in(x,z) \land on(z,y) \land top(x))) \lor (\lnot on(x,y) \land \exists z (Stack(z) \land in(x,z) \land on(z,y) \land top(x)))) \Rightarrow over(x,y))$

    \item $\forall x ((Block(x) \land rot(x)) \Rightarrow in (x,1))$

    \item $\forall x (in(x,2) \Rightarrow (Block(x) \land blau(x)))$
\end{enumerate}

\section*{Aufgabe 2}

\begin{enumerate}
    \item Implikationen entfernen, denn in einer KNF kommen lediglich Disjunktionen und Konjunktionen vor. Wir verwenden die Auflösung für Implikationen ($a \Rightarrow b \equiv \lnot a \lor b$):\\
    $(\forall x)(\lnot P(x) \lor ((\forall y)(\lnot P(y) \lor P(f(x,y))) \land \lnot (\forall y)(\lnot Q(x,y) \lor P(y))))$

    \item Negationen nach innen bringen, um nur Literale zu haben. Wir verwenden die Regel von De'Morgan:\\
    $(\forall x)(\lnot P(x) \lor ((\forall y)(\lnot P(y) \lor p(f(x,y))) \land (\exists y)(Q(x,y) \land \lnot P(y))))$

    \item Unabhängige Variablen (solche, die im Bindungsbereich verschiedener Quantoren liegen) umbenennen, um Eindeutigkeit herzustellen, damit später ohne Probleme Quantoren nach außen gezogen werden können:\\
    $(\forall x)(\lnot P(x) \lor ((\forall y)(\lnot P(y) \lor P(f(x,y))) \land (\exists z)(Q(x,z) \land \lnot P(z))))$

    \item Um eine KNF zu erhalten, müssen wir die Quantoren entfernen. Zuerst bringen wir sie dafür nach außen:\\
    $(\forall x)(\forall y)(\exists z)(\lnot P(x) \lor ((\lnot P(y) \lor P(f(x,y))) \land (Q(x,z) \land \lnot P(z))))$

    \item Quantoren entfernen, indem wir Existenzquantoren durch Konstanten oder n-stellige Funktionen ersetzen, wobei n die Anzahl der Allquantoren ist, in deren Scope sich der Existenzquantor befindet. Danach können die Allquantoren einfach weggelassen werden, da dann alle Variablen implizit allquantisiert sind. Wir erhalten eine erfüllbarkeitsäquivalente skolemisierte Formel. Wir führen also für z eine 2-stellige Funktion g(x,y) ein:\\
    $\lnot P(x) \lor ((\lnot P(y) \lor P(f(x,y))) \land (Q(x,g(x,y)) \land \lnot P(g(x,y))))$

    \item Nach KNF umformen (in diesem Fall durch iterative Anwendung des Distributivgesetzes):\\
    $(\lnot P(x) \lor \lnot P(y) \lor P(f(x,y))) \land (\lnot P(x) \lor Q(x,g(x,y))) \land (\lnot P(x) \lor \lnot P(g(x,y)))$\\
    In Mengenschreibweise:\\
    $\{\{\lnot P(x), \lnot P(y), P(f(x,y))\}, \{\lnot P(x),Q(x,g(x,y))\},\{\lnot P(x),\lnot P(g(x,y))\}\}$ \\
    Die Klausel sollten jeweils paarweise Disjunkte Variablen enthalten, damit beispielsweise die Resolution einfach durchgeführt werden kann. Dafür kann man die Variablen umbenennen:\\
    $\{\{\lnot P(x), \lnot P(y), P(f(x,y))\}, \{\lnot P(a),Q(a,g(a,b))\},\{\lnot P(c),\lnot P(g(c,d))\}\}$
\end{enumerate}

\section*{Aufgabe 3}
Wir schreiben für jede Iteration jeweils $\theta$ und $L\theta$ auf. Die unterstrichenen Teile sind jeweils die nächsten Stellen an denen sich 2 Klauseln unterscheiden.

\begin{enumerate}[~~a)]
    \item
    \begin{enumerate}[~~1.]
      \item $\theta = \{\}$\\
            $L\theta = \{ \underline{P(x, g(y), g(g(z)))},~ P(u, u, u),~ \underline{v} \} $
      \item $\theta = \{ v/P(x, g(y), g(g(z))) \}$\\
            $L\theta = \{ P(\underline{x}, g(y), g(g(z))),~ P(\underline{u}, u, u)\} $
      \item $\theta = \{ v/P(x, g(y), g(g(z))),~ x/u \}$\\
            $L\theta = \{ P(u, \underline{g(y)}, g(g(z))),~ P(u, \underline{u}, u)\} $
      \item $\theta = \{ v/P(x, g(y), g(g(z))),~ x/u,~ u/g(y) \}$\\
            $L\theta = \{ P(g(y), g(y), g(g(\underline{z}))),~ P(g(y), g(y), g(\underline{y})))\} $
      \item $\theta = \{ v/P(x, g(y), g(g(z))),~ x/u,~ u/g(y),~ y/g(z)\}$\\
            $L\theta = \{ P(g(g(z)), g(g(z)), g(g(z)))\} $
    \end{enumerate}

    \item
    \begin{enumerate}[~~1.]
      \item $\theta = \{\}$\\
            $L\theta = \{ P(\underline{x}, y, f(w, z)),~ P(\underline{1}, v, f(1, 2, 3)) \}$
      \item $\theta = \{x/1\}$\\
            $L\theta = \{ P(1, \underline{y}, f(w, z)),~ P(1, \underline{v}, f(1, 2, 3)) \}$
      \item $\theta = \{x/1,~ y/v\}$\\
            $L\theta = \{ P(1, v, \underline{f(w, z)}),~ P(1, v, \underline{f(1, 2, 3)}) \}$ \\
            Abbruch, nicht unifizierbar, die Funktionen nehmen unterschiedlich viele Argumente.

    \end{enumerate}

    \item
    \begin{enumerate}[~~1.]
      \item $\theta = \{\}$\\
            $L\theta = \{ P(\underline{f(x, y)}, f(y, x)),~ P(\underline{z}, x) \}$ 
      \item $\theta = \{z/f(x,y)\}$\\
            $L\theta = \{ P(f(x, y), \underline{f(y, x))},~ P(f(x,y), \underline{x}) \}$ \\
            Abbruch, $x$ kommt in $f(y,x)$ vor.

    \end{enumerate}

    \item
    \begin{enumerate}[~~1.]
      \item $\theta = \{\}$\\
            $L\theta = \{ P(\underline{x}, g(v, v), b),~ P(\underline{f(y, z)}, g(x, w), h(1, z)) \}$
      \item $\theta = \{x/f(y,z)\}$\\
            $L\theta = \{ P(f(y,z), g(\underline{v}, v), b),~ P(f(y, z), g(\underline{x}, w), h(1, z)) \}$
      \item $\theta = \{x/f(y,z), ~ v/x\}$\\
            $L\theta = \{ P(f(y,z), g(x, \underline{x}), b),~ P(f(y, z), g(x, \underline{w}), h(1, z)) \}$
      \item $\theta = \{x/f(y,z), ~ v/x, ~ x/w\}$\\
            $L\theta = \{ P(f(y,z), g(w, w), \underline{b}),~ P(f(y, z), g(w, w), \underline{h(1, z)}) \}$
      \item $\theta = \{x/f(y,z), ~ v/x, ~ x/w, ~ b/h(1,z)\}$\\
            $L\theta = \{ P(f(y,z), g(w, w), h(1,z)) \}$
    \end{enumerate}
\end{enumerate}

\section*{Aufgabe 4}

\begin{enumerate}[~~a)]
    \item
Die Formeln in KNF:
\begin{align*}
&(1) \lnot Be(x,GKI) \lor \lnot Ge(x, Poker) \lor Fr(x)\\
&(2) (\lnot Fl(x) \lor Be(x,y)) \land (\lnot In(x) \lor Be(x,y))\\
&(3) \lnot Be(x,y) \lor Te(x,f(x,y))\\
&(4) \lnot In(x) \lor Ge(x, Poker)\\
&(5) \& (6) \textit{ sind bereits in KNF}
\end{align*}
Die Klauseln Mengenschreibweise mit umbenannten Variablen (nummeriert für die nächste Teilaufgabe):
\begin{align*}
(1) &\{\lnot Be(x, GKI), \lnot Ge(x,Poker) , Fr(x) \}\\
(2) &\{\lnot Fl(a), Be(a,b)\}\\
(3) &\{ \lnot In(c), Be(c,d)\}\\
(4) &\{\lnot Be(e,g), Te(e,f(e,g))\}\\
(5) &\{\lnot In(h), Ge(h, Poker)\}\\
(6) &\{\lnot Fl(O)\}\\
(7) &\{In(O)\}\\
(8) &\{Fl(S)\}\\
(9) &\{\lnot In(S)\}
\end{align*}
    \item
Um zu prüfen, wer froh ist, stellen wir folgende Formel auf:
\begin{align*}
\exists y(Fr(y))
\end{align*}
Diese Formel drückt aus, dass es jemanden gibt, der froh ist. Wir wollen nun testen:
\begin{align*}
KB \vDash \exists y (Fr(y))
\end{align*}
Um diese Inferenz mittels Resolution beweisen zu können, müssen wir die negierte Anfrage als Klausel zur Wissensbasis hinzufügen und einen Widerspruch herleiten (wir verwenden die Klauseln und Nummerierung aus 4 a) und fahren nun hier direkt mit der negierten Anfrage als Klause fort):
\begin{align*}
(10) &\{\lnot Fr(y)\}
\end{align*}
Resolution:
\begin{align*}
(11) &\{\lnot Be(y, GKI), \lnot Ge(y,Poker)\} &&\dashv (1) \& (10), \{x/y\}\\
(12) &\{\lnot In(y),\lnot Ge(y,Poker)\} &&\dashv (3) \& (11), \{x/y,d/Poker\}\\
(13) &\{\lnot In(y)\} &&\dashv (5) \& (12), \{h/y\}\\
(14) &\{\} && \dashv (7) \& (13), \{y/O\}\\
\end{align*}
Da wir mit unserer Anfrage einen Widerspruch herleiten konnten, haben wir somit gefolgert, dass es jemanden gibt, der froh ist. Wenn man die Substitutionen betrachtet, sieht man, dass es Olli ist (wegen der letzten Substitution $\{y/O\}$ muss gelten $Fr(O)$) und somit gilt auch:
\begin{align*}
KB \vDash \exists y (Fr(y))
\end{align*}
    \item
Wie man in 4 a) sehen kann, lässt sich Formel (2) vom Aufgabenblatt in zwei Klauseln umformen. Somit können wir einfach zwei Regeln als definite Hornformeln aufstellen, die (2) vom Aufgabenblatt in unserer Hornklausel-Wissensbasis repräsentieren:
\begin{align*}
Fl(x) \Rightarrow Be(x,y)\\
In (x) \Rightarrow Be(x,y)
\end{align*}
Die negativen Literale aus (5) und (6) können wir einfach ignorieren, da sie für die Inferenzregel, welche wir bei der Vorwärts- bzw. Rückwärtsverkettung benutzen (den generalisierten Modus Ponens), irrelevant sind. Denn wir können mit dieser Regel, die für definite Hornklausel-Wissensbasen vollständig ist, lediglich aus Fakten ('unnegiertem' Wissen) neues Wissen folgern.
Also ergibt sich mit diesen Überlegungen insgesamt folgende Hornklausel-Wissensbasis (die Variablen haben wir für die Formeln paarweise disjunkt benannt):
\begin{align*}
(1) &Be(x,GKI)\land Ge(x,Poker) \Rightarrow Fr(x)\\
(2) &Fl(y) \Rightarrow Be(y,z)\\
(3) &In (a) \Rightarrow Be(a,b)\\
(4) &Be(c,d) \Rightarrow Te(c,f(c,d))\\
(5) &In(e) \Rightarrow Ge(e,Poker)\\
(6) &In(O)\\
(7) &Fl(S)
\end{align*}
    \item
\begin{align*}
(1)&[Fr(O),Te(S,f(S,GKI)),Te(O,f(O,GKI)),Be(S,GKI),Be(O,GKI),Ge(O,Poker)]\\
(2)&[Be(O,GKI),Ge(O,Poker),Te(S,f(S,GKI)),Te(O,f(O,GKI)),Be(S,GKI)] &&\dashv (1) \{x/GKI\} \\
(3)&[Be(O,GKI),Ge(O,Poker),Be(S,GKI),Te(O,f(O,GKI))] &&\dashv (4) \{c/S,d/GKI\} \\
(4)&[Ge(O,Poker),Be(S,GKI),Be(O,GKI)] &&\dashv (4) \{c/O,d/GKI\} \\
(5)&[In(O),Be(S,GKI),Be(O,GKI)] &&\dashv (5) \{e/O\} \\
(6)&[Be(S,GKI),In(O)] &&\dashv (3) \{a/O,b/GKI\} \\
(7)&[Be(S,GKI)] &&\dashv (6) \\
(8)&[Fl(S)] &&\dashv (2) \{y/S,z/GKI\} \\
(9)&[] &&\dashv (7) \\
\end{align*}
    \item
\end{enumerate}

\section*{Aufgabe 5}
Ursprüngliche Klauseln:\\
(1) $\{P(f(w)), Q(f(x))\}$\\
(2) $\{ \lnot P(f(1)), \lnot P(y) \}$\\
(3) $\{Q(g(z)), Q(g(f(f(z))))\}$\\
(4) $\{\lnot Q(f(g(1))), \lnot Q(g(f(v))), \lnot Q(g(v))\}$\\
\\
Resolutionsbeweis:\\
\begin{align*}
(5) &\{\lnot P(y), Q(f(x))\} &&\dashv (1) \& (2), \{w/1\}\\
(6) &\{Q(f(x))\} &&\dashv (1) \& (5), \{y/f(w)\}\\
(7) &\{\lnot Q(f(g(1))), \lnot Q(g(f(a))), Q(g(f(f(a))))\} &&\dashv (3) \& 		(4), \{z/v,v/a\}\\
(8) &\{\lnot Q(f(g(1))), \lnot Q(g(f(b)))\} &&\dashv (4) \& (7), \{v/f(a), 		a/b\}\\
(9) &\{\lnot Q(g(f(b)))\} &&\dashv (6) \& (8), \{x/g(1)\}\\
(10) &\{Q(g(z))\} &&\dashv (3) \& (9), \{b/f(z)\}\\
(11) &\{\} &&\dashv (9) \& (10), \{z/f(b)\}\\
\end{align*}
\end{document}
