\documentclass[a4paper,10pt]{article}
\usepackage[utf8]{inputenc}
\usepackage{amsmath}
\usepackage{amsfonts}
\usepackage{amssymb}
\usepackage[german]{babel}
\setlength{\parindent}{0cm}
\usepackage{setspace}
\usepackage{mathpazo}
\usepackage{listings}
\usepackage{graphicx}
\usepackage{wasysym}
\usepackage{booktabs}
\usepackage{verbatim}
\usepackage{ulem}
\usepackage{enumerate}
\usepackage{hyperref}
\usepackage{ulem}
\usepackage{stmaryrd }
\usepackage[a4paper,
left=1.8cm, right=1.8cm,
top=2.0cm, bottom=2.0cm]{geometry}
\usepackage{tabularx}
\usepackage{tikz}
\usepackage{cancel}
\usetikzlibrary{trees,petri,decorations,arrows,automata,shapes,shadows,positioning,plotmarks}

\newcommand{\rf}{\right\rfloor}
\newcommand{\lf}{\left\lfloor}
\newcommand{\tabspace}{15cm}
\newcommand{\N}{\mathbb{N}}
\newcommand{\Z}{\mathbb{Z}}

\begin{document}
\begin{center}
\Large{Grundlagen der künstlichen Intelligenz: Hausaufgabe 6} \\
\end{center}
\begin{tabbing}
Tom Nick \hspace{2cm}\= - 340528\\
Niklas Gebauer \> - 340942 \\
Leonard Witte \> - 341457 \\
Johannes Herrmann \> - 341091\\
\end{tabbing}

\section*{Aufgabe 1 - Hidden Markov-Prozess}
Die Anfangsbedingung wird mit $P(X_0 = w) = 0.5$ angenommen.
\begin{enumerate}[~~(a)]
    \item \begin{align*}
    P(X_1 = \dots = X_k = w, X_{k+1} = f \mid X_0 = f) &= P(X_{k+1} = f \mid X_{k} = w) \cdot \prod_{i = 1}^{k}P(X_{i+1} = w \mid X_i = w) \cdot P(X_1 = w \mid X_0 = f)\\
    &= 0.1 \cdot 0.8^{k-1} \cdot 0.2
    \end{align*}
    \item \begin{align*}
        p_t &= P(X_t = w \mid Y_{1:t}) \\
        p_{t-1} &= P(X_{t-1} = w \mid Y_{1:t-1}) \\
        P(X_t = w \mid Y_1, ... , Y_t = g) &= \alpha P(Y_t =g \mid X_t = w) \sum_{X_{t-1}} P(X_t = w \mid X_{t-1}) P(X_{t-1} \mid Y_{1:t-1})\\
        &= \alpha * 0,3 * (0,8*p_{t-1} + 0,1*(1-p_{t-1}))\\
        &= \alpha * 0,24*p_{t-1} + \alpha * 0,03 - \alpha * 0,03*p_{t-1} \\
        &= \alpha * (0,21*p_{t-1} + 0,03)\\
\\
        \alpha &= \frac{1}{P(Y_t = g \mid Y_{1:t-1})}
    \end{align*}
\begin{comment}
        &= \alpha p(X_t, Y_t \mid Y_{1:t-1}) = \alpha p(Y_{t} \mid X_{t}, Y_{1:t-1})p(x_{t}, e_{1:t-1}) \\
        &= \alpha p(Y_t, X_t)p(X_t, Y_{1:t-1}) \\
        p(X_t\mid Y_{1:t-1}) &= \sum_{x_t} p(X_{t}, X_t \mid Y_{1:t-1}) = \sum_{x_t}p(X_t\mid X_{t -1})p(X_t\mid Y_{1:t}) \\
        p(X_t \mid Y_{1:t}) &= \alpha p(Y_t \mid X_t) \sum_{x_t} p(X_t\mid X_{t-1})p(X_t\mid Y_{1:t-1}) \\
        &= \alpha p(Y_t = g \mid X_t = w) \sum_{x_t} p(X_t\mid X_{t-1})p(X_t\mid Y_{1:t-1}) \\
        &= \alpha 0.3 \cdot \sum_{x_t} p(X_t\mid X_{t-1})p(X_t\mid Y_{1:t-1})
    \end{align*}
\end{comment}
    \item \begin{align*}
        P(X_1 = w) &= P(X_1 = w|X_0 = f) P(x_0 =f) + P(X_1 = w|x_0 = w) P(X_0 = w)\\
        &= 0.1 \cdot 0.5 + 0.2 \cdot 0.5) = 0.45\\
        P(X_1 = f) &= P(X_1 = f|X_0 = f) P(x_0 =f) + P(X_1 = f|x_0 = w) P(X_0 = w)\\
        &= 0.9 \cdot 0.5 + 0.2 \cdot 0.5 = 0.45\\
        P(X_1 = w|Y_1 = c) &= \frac{P(X_1 = w, Y_1 = c)}{P(Y_1 = c)}\\
        &= \frac{P(X_1 = w| Y_1 = c)P(Y_1 = c)}{P(Y_1 = c|X_1 = w)P(X_1 = w)+P(Y_1 = c|X_1 = f)P(X_1 = f)}\\
        &= \frac{0.3 \cdot 0.45}{0.3 \cdot 0.45 + 0.2 \cdot 0.55} = \frac{0.135}{0.245}\\
        P(X_1 = f|Y_1 = c) &= \frac{P(X_1 =f,y_1 = c)}{P(Y_1 = c)} = \frac{0.2 \cdot 0.55}{0.3 \cdot 0.45 + 0.2 \cdot 0.55} = \frac{0.11}{0.245}\\
        P(Y_2 = g|Y_1 = c) &= \sum\limits_{X_2} [P(Y_2 = g|X_2) P(X_2|Y_1 = c)]\\
        &= \sum\limits_{X_2} [P(Y_2 = g) \sum\limits_{X_1} [P(X_2|X_1)P(X_1|Y_1 = c)]]\\
        &= 0.3(0.8 \frac{0.135}{0.245} + 0.1 \frac{0.11}{0.245}) + 0.2(0.2 \frac{0.135}{0.245} + 0.9 \frac {0.11}{0.245}) = \frac{0.0609}{0.245}\\
        P(X_2 = w|Y_2 = g,Y_1 = c) &= \frac{P(X_2 = w, Y_2 = g, Y_1 = c)}{P(Y_2 = g, Y_1 = c)}\\
        &= \frac{P(Y_2 = g|X_2 = w) P(X_2 = w|Y_1 =c)\cancel{P(Y_1 = c)}}{P(Y_2 = g|Y_1 = c)\cancel{P(Y_1 = c)}}\\
        &= \frac{P(Y_2 = g|X_2 = w) \sum\limits_{X_1} [P(X_2 = w|X_1)P(X_1|Y_1 = c)]}{P(Y_2 = g|Y_1 = c)}\\
        &= \frac{0.3(0.8 \frac{0.135}{0.245} + 0.1 \frac{0.11}{0.245})}{\frac{0.0609}{0.245}}\\
        &= \frac{\frac{0.0357}{\cancel{0.245}}}{\frac{0.0609}{\cancel{0.245}}} = \frac{0.0357}{0.0609}\\
        P(X_2 = f|Y_2 = g,Y_1 = c) &=\frac{0.2(0.2 \frac{0.135}{0.245} + 0.9 \frac {0.11}{0.245})}{\frac{0.0609}{0.245}}\\
        &=\frac{\frac{0.0252}{\cancel{0.245}}}{\frac{0.0609}{\cancel{0.245}}} = \frac{0.0252}{0.0609}\\
        P(X_3 = w| Y_2 = g, Y_1 = c) &= \sum\limits_{X_2} [P(X_3 = w|X_2) P(X_2|Y_2 = g, Y_1 = c)]\\
        &= 0.8 \frac{0.0357}{0.0609} + 0.1 \frac{0.0252}{0.0609} = \frac{0.03108}{0.0609} \approx 0.51\\
        P(X_3 = f| Y_2 = g, Y_1 = c) &= \sum\limits_{X_2} [P(X_3 = f|X_2) P(X_2|Y_2 = g, Y_1 = c)]\\
        &= 0.2 \frac{0.0357}{0.0609} + 0.9 \frac{0.0252}{0.0609} = \frac{0.02982}{0.0609}\\
        P(X_4 = w| Y_2 = g, Y_1 = c) &= \sum\limits_{X_3} [P(X_4 = w|X_3) P(X_3|Y_2 = g, Y_1 = c)]\\
        &= 0.8 \frac{0.03108}{0.0609} + 0.1 \frac{0.02982}{0.0609} = \frac{0.027846}{0.0609} \approx 0.46\\
    \end{align*}
    \item
    \begin{align*}
        P(X_1 = w| Y_1 = c, Y_2 = g) &= \frac{P(Y_2 = g|X_1 = w)P(X_1 = w|Y_1 = c)}{\sum\limits_{X_1} [P(Y_2 = g|X_1) P(X_1|Y_1 = c)]}\\
        &= \frac{\sum\limits_{X_2} [P(X_2|X_1 = w)P(Y_2 = g|X_2))P(X_1 = w|y_1 = c)]}{\sum\limits_{X_1} [\sum\limits_{X_2} [P(X_2|X_1)P(Y_2 = g|X_2)]P(X_1|Y_1 = c)]}\\
        &= \frac{(0.8 \cdot 0.3 + 0.2 \cdot 0.2) \frac{0.135}{0.245}}{((0.8 \cdot 0.3 + 0.2 \cdot 0.2) \frac{0.135}{0.245}) + ((0.1 \cdot 0.3 + 0.9 \cdot 0.2) \frac{0.11}{0.245})}\\
        &\approx 0.62\\
    \end{align*}
    \item
	\begin{align*}
		m(w,0) &= P(X_0 = w) = 0,5 \\
		m(f,0) &= P(X_0 = f) = 0,5\\ \\
		m(w,1) &= P(Y_1 = a \mid X_1 = w) * max_v (P(X_1 = w \mid X_0 = v) * m(v,0)) \\
		&= 0,2 * max[0,8*0,5;0,1*0,5]\\
		&= 0,08  (mit X_0 = w)\\
		m(f,1) &= P(Y_1 = a \mid X_1 = f) * max_v (P(X_1 = f \mid X_0 = v) * m(v,0)) \\
		&= 0,3 * max[0,2*0,5;0,9*0,5]\\
		&= 0,135  (mit X_0 = f)\\ \\
		m(w,2) &= P(Y_2 = c \mid X_2 = w) * max_v (P(X_2 = w \mid X_1 = v) * m(v,1)) \\
		&= 0,3 * max[0,8*0,08;0,1*0,135]\\
		&= 0,0192  (mit X_1 = w)\\
		m(f,2) &= P(Y_2 = a \mid X_2 = f) * max_v (P(X_2 = f \mid X_1 = v) * m(v,1)) \\
		&= 0,2 * max[0,2*0,08;0,9*0,135]\\
		&= 0,0243  (mit X_1 = f)\\ \\
		m(w,3) &= P(Y_3 = g \mid X_3 = w) * max_v (P(X_3 = w \mid X_2 = v) * m(v,2)) \\
		&= 0,3 * max[0,8*0,0192;0,1*0,0243]\\
		&= 0,004608  (mit X_2 = w)\\
		m(f,3) &= P(Y_3 = g \mid X_3 = f) * max_v (P(X_3 = f \mid X_2 = v) * m(v,2)) \\
		&= 0,2 * max[0,2*0,0192;0,9*0,0243]\\
		&= 0,004374  (mit X_2 = f)\\ \\
		m(w,4) &= P(Y_4 = t \mid X_4 = w) * max_v (P(X_4 = w \mid X_3 = v) * m(v,3)) \\
		&= 0,3 * max[0,8*0,004608;0,1*0,004374]\\
		&= 0,00073728  (mit X_3 = w)\\
		m(f,4) &= P(Y_4 = t \mid X_4 = f) * max_v (P(X_4 = f \mid X_3 = v) * m(v,3)) \\
		&= 0,2 * max[0,2*0,004608;0,9*0,004374]\\
		&= 0,00118098  (mit X_3 = f)\\ \\
	\end{align*}
Es folgt also:
$X_4 = f, X_3 = f, X_2 = f, X_1 = f, X_0 = f$
\end{enumerate}

\section*{Aufgabe 2 - Hidden Markov-Modell}
\begin{enumerate}[~~(a)]
    \item ~\\
    \begin{center}
        \begin{tikzpicture}[auto,bend angle=30,node distance=1.5cm]
            % Zustaende
            % \node[state] (1) {(};
            % \node[state] (2)[right of=1] {x};
            % \node[state] (3)[right of=2] {y};
            % \node[state] (4)[right of=3] {z};
            % \node[state] (5)[right of=4] {0};
            % \node[state] (6)[right of=5] {1};
            % \node[state] (7)[right of=6] {2};
            % \node[state] (8)[right of=7] {3};
            % \node[state] (9)[right of=8] {+};
            % \node[state] (10)[right of=9] {-};
            % \node[state] (11)[right of=10] {)};

            \node[state] (12)[] {$k_1$};
            \node[state] (13)[right of=12] {$v$};
            \node[state] (14)[right of=13] {$z_1$};
            \node[state] (15)[right of=14] {$z_2$};
            \node[state] (16)[right of=15] {$rs$};
            \node[state] (17)[right of=16] {$k_2$};
            \node[state] (18)[above of=12] {start};
            \node[state] (19)[above of=17] {end};

            \path[->] (18) edge  (12);
            \path[->] (17) edge  (19);

            % \path[->] (13) edge  (2);
            % \path[->] (13) edge  (3);
            % \path[->] (13) edge  (4);
            
            % \path[->] (14) edge  (6);
            % \path[->] (14) edge  (7);
            % \path[->] (14) edge  (8);

            % \path[->] (15) edge  (5);
            % \path[->] (15) edge  (6);
            % \path[->] (15) edge  (7);
            % \path[->] (15) edge  (8);

            % \path[->] (16) edge  (9);
            % \path[->] (16) edge  (10);

            % \path[->] (17) edge  (11);
            
            \path[->] (12) edge  (13);
            \path[->, bend right] (12) edge  (14);

            \path[->, bend right] (13) edge  (16);
            \path[->, bend right] (13) edge  (17);

            \path[->, bend right] (14) edge  (15);
            \path[->, bend right] (14) edge  (16);
            \path[->, bend right] (14) edge  (17);

            \path[->, loop right] (15) edge  (15);
            \path[->, bend left] (15) edge  (16);
            \path[->, bend right] (15) edge  (17);
            
            \path[->, bend right] (16) edge  (14);
            \path[->, bend right] (16) edge  (13);


        \end{tikzpicture}
    \end{center}
    \item
    \begin{align*}
        P(x_{0:6} \mid y_{1:6} = (x+30)) &= P(x_0)\prod_{t=1}^{6}P(y_t\mid x_t)P(x_t\mid x_{t - 1}) \\
        &= 1.0  \cdot (0.6 \cdot 0.6) \cdot (0.8 \cdot 0.7) \cdot (0.7 \cdot 0.2) \cdot (0.5 \cdot 0.4) \cdot (0.3 \cdot 1.0) \cdot 1.0 \\
        &= 0.000169344
    \end{align*}

    \item Es gibt folgende Möglichkeiten:
    \begin{itemize}
        \item $\dots ~rs~ \{x, y, z\} ~rs~ \dots$
        \item $\dots ~rs~ \{1, 2 ,3\} ~rs~ \dots$
        \begin{align*}
            P(~rs~ x ~rs~) &= P(v \mid rs) \cdot P(x \mid v) \cdot P(rs \mid v) = 0.3 \cdot 0.6 \cdot 0.8 = 0.144 \\
            P(~rs~ y ~rs~) &= 0.3 \cdot 0.3 \cdot 0.8 = 0.072 \\
            P(~rs~ z ~rs~) &= 0.3 \cdot 0.1 \cdot 0.8 = 0.024 \\
            P(~rs~ 1 ~rs~) &= 0.7 \cdot 0.5 \cdot 0.3 = 0.105 \\
            P(~rs~ 2 ~rs~) &= 0.7 \cdot 0.3 \cdot 0.3 = 0.063 \\
            P(~rs~ 3 ~rs~) &= 0.7 \cdot 0.2 \cdot 0.3 = 0.042 \\
        \end{align*}

        Somit ist die Variable $x$ am wahrscheinlichsten. Die Aussage sollte mit einer Wahrscheinlichkeit von $\frac{0.144}{0.144 + 0.072 + 0.024 + 0.105 + 0.063 + 0.042} = 0.32$ zutreffen.

    \end{itemize}
    \item Die Möglichkeiten wären $(z_1, z_2) z_1 \in \{0,1,2\}, z_2 \in \{0,1,2,3\}$ bzw. $start~k_1 z_1 z_2 k_2~stop$
    $$P(x_5 = sto,x_4 = k_2, x_3 = z_2, x_2 = z_1, x_1 = k_1, x_0 = sta) = 1.0 \cdot 0.4 \cdot 0.5 \cdot 0.3 \cdot 1.0 = 0.06$$
    6\% ist die Wahrscheinlichkeit das ein mathematischer Ausdruck mit genau 4 Zeichen auftritt.
\end{enumerate}


\end{document}
